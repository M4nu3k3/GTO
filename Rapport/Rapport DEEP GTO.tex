\documentclass[a4paper,11pt]{article}
\usepackage[utf8]{inputenc}
\usepackage[T1]{fontenc}
\usepackage[french]{babel}
\usepackage{geometry}
\geometry{margin=2cm}

\begin{document}

\tableofcontents
\newpage

\section{Introduction}
\begin{itemize}
    \item Contexte général (évaluations papier automatisées)
    \item Objectifs du projet
    \begin{itemize}
        \item Génération automatique de QCM scannables
        \item Correction automatique par traitement d'image
        \item Affichage automatique des résultats
    \end{itemize}
    \item Outils utilisés (Python, Turtle, STM32, OpenCV, MATLAB)
\end{itemize}

\section{Cahier des charges fonctionnel et technique}

\subsection{Environnement d'utilisation}
\begin{itemize}
    \item Température : 10 à 30°C
    \item Hygrométrie relative : 40 à 95\%
\end{itemize}

\subsection{Alimentation}
\begin{itemize}
    \item Alimentation AC/DC : 5V, 1A
    \item Consommation maximale : 12W
\end{itemize}

\subsection{Entrées}
\begin{itemize}
    \item Adaptateur SD (lecture fichiers QCM)
    \item Carte micro-SD (stockage fichiers QCM)
    \item Entrées numériques : 3 entrées (0-3.3V)
\end{itemize}

\subsection{Sorties}
\begin{itemize}
    \item Écran OLED (sélection fichiers, affichage résultats)
    \item LED témoin (traitement en cours)
\end{itemize}

\subsection{Interfaces Homme-Machine}
\begin{itemize}
    \item Écran OLED (résultats : élève, score)
    \item Codeur incrémental (navigation)
    \item LED de statut
\end{itemize}

\section{Conception et architecture du projet}

\subsection{Schéma Bloc}
Présentation visuelle des blocs fonctionnels : STM32, SD, OLED, LED, codeur incrémental, alimentation.

\subsection{Choix techniques}
Justification des choix techniques (STM32, OLED, carte SD, etc.)

\section{Génération automatique des QCM (Python/Turtle)}

\subsection{Structure visuelle des feuilles}
\begin{itemize}
    \item Format A4, 2 colonnes, 20 questions
    \item Bandeau d'identification ID élève
    \item Cercles réponses (A/B)
    \item Simulation remplissage réaliste
\end{itemize}

\subsection{Génération des fichiers associés}
\begin{itemize}
    \item Export PDF via \texttt{epstopdf}
    \item Génération coordonnées cases (format .gto/json)
\end{itemize}

\section{Traitement d'image}

\subsection{Prétraitement (MATLAB/OpenCV)}
\begin{itemize}
    \item Chargement images scannées
    \item Conversion niveaux de gris
    \item Seuillage adaptatif, opérations morphologiques
\end{itemize}

\subsection{Extraction des réponses}
\begin{itemize}
    \item Détection réponses cochées
    \item Validation positions (bounding boxes)
\end{itemize}

\subsection{Comparaison et notation automatique}
\begin{itemize}
    \item Chargement corrigé (.gto/json)
    \item Comparaison réponses élève/corrigé
    \item Calcul et affichage notes (OLED)
\end{itemize}

\section{Interfaces et interactions}

\subsection{Écran OLED}
Sélection fichier, affichage résultats

\subsection{Codeur incrémental}
Navigation intuitive, validation

\subsection{LED}
Indicateur traitement en cours

\section{Résultats préliminaires et validation}
Performance actuelle (temps traitement, précision)

\section{Difficultés rencontrées et solutions apportées}
Alignement graphique, précision traitement image, conversion coordonnées

\section{Perspectives d'amélioration}
\begin{itemize}
    \item QR code identification
    \item Traitement par lots
    \item Interface utilisateur avancée (Tkinter/webapp)
    \item Reconnaissance OCR pour ID
\end{itemize}

\section{Conclusion}
Résumé objectifs atteints, bilan état actuel, prochaines étapes

\appendix

\section{Annexes}
\begin{itemize}
    \item Extraits code Python (Turtle)
    \item Extraits code MATLAB/OpenCV
    \item Schéma bloc détaillé
    \item Tableau spécifications techniques
    \item Captures d'écran résultats tests
    \item Arborescence projet
\end{itemize}

\end{document}
